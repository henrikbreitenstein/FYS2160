\documentclass[reprint,english,notitlepage]{revtex4-1}  % defines the basic parameters of the document

% if you want a single-column, remove reprint

% allows special characters (including æøå)
\usepackage[utf8]{inputenc}
\usepackage[english]{babel}

%% note that you may need to download some of these packages manually, it depends on your setup.
%% I recommend downloading TeXMaker, because it includes a large library of the most common packages.

\usepackage{physics,amssymb}  % mathematical symbols (physics imports amsmath)
\usepackage{graphicx}         % include graphics such as plots
\usepackage{xcolor}           % set colors
\usepackage{hyperref}         % automagic cross-referencing (this is GODLIKE)
\usepackage{tikz}             % draw figures manually
\usepackage{listings}         % display code
\usepackage{subfigure}        % imports a lot of cool and useful figure commands

% defines the color of hyperref objects
% Blending two colors:  blue!80!black  =  80% blue and 20% black
\hypersetup{ % this is just my personal choice, feel free to change things
    colorlinks,
    linkcolor={red!50!black},
    citecolor={blue!50!black},
    urlcolor={blue!80!black}}

%% Defines the style of the programming listing
%% This is actually my personal template, go ahead and change stuff if you want
\lstset{ %
	inputpath=,
	backgroundcolor=\color{white!88!black},
	basicstyle={\ttfamily\scriptsize},
	commentstyle=\color{magenta},
	language=Python,
	morekeywords={True,False},
	tabsize=4,
	stringstyle=\color{green!55!black},
	frame=single,
	keywordstyle=\color{blue},
	showstringspaces=false,
	columns=fullflexible,
	keepspaces=true}


%% USEFUL LINKS:
%%
%%   UiO LaTeX guides:        https://www.mn.uio.no/ifi/tjenester/it/hjelp/latex/ 
%%   mathematics:             https://en.wikibooks.org/wiki/LaTeX/Mathematics

%%   PHYSICS !                https://mirror.hmc.edu/ctan/macros/latex/contrib/physics/physics.pdf

%%   the basics of Tikz:       https://en.wikibooks.org/wiki/LaTeX/PGF/TikZ
%%   all the colors!:          https://en.wikibooks.org/wiki/LaTeX/Colors
%%   how to draw tables:       https://en.wikibooks.org/wiki/LaTeX/Tables
%%   code listing styles:      https://en.wikibooks.org/wiki/LaTeX/Source_Code_Listings
%%   \includegraphics          https://en.wikibooks.org/wiki/LaTeX/Importing_Graphics
%%   learn more about figures  https://en.wikibooks.org/wiki/LaTeX/Floats,_Figures_and_Captions
%%   automagic bibliography:   https://en.wikibooks.org/wiki/LaTeX/Bibliography_Management  (this one is kinda difficult the first time)
%%   REVTeX Guide:             http://www.physics.csbsju.edu/370/papers/Journal_Style_Manuals/auguide4-1.pdf
%%
%%   (this document is of class "revtex4-1", the REVTeX Guide explains how the class works)


%% CREATING THE .pdf FILE USING LINUX IN THE TERMINAL
%% 
%% [terminal]$ pdflatex template.tex
%%
%% Run the command twice, always.
%% If you want to use \footnote, you need to run these commands (IN THIS SPECIFIC ORDER)
%% 
%% [terminal]$ pdflatex template.tex
%% [terminal]$ bibtex template
%% [terminal]$ pdflatex template.tex
%% [terminal]$ pdflatex template.tex
%%
%% Don't ask me why, I don't know.

\begin{document}
\title{Equations}   % self-explanatory
\author{Henrik Modahl Breitenstein}               % self-explanatory
\date{\today}                             % self-explanatory
\noaffiliation                            % ignore this                           % marks the end of the abstract
\maketitle

\section*{First Law}

Energy is always conserved. It can not be made or disappear for a closed system.

\begin{equation}
\label{first}
\Delta{U} = Q + W \; .
\end{equation}

Where $U$ is the internal energy, $Q$ is the heat and $W$ is the work.

\section*{Second Law}

For a isolated system the change in entropy is never below zero.

\begin{equation}
\label{second}
\Delta{S} \geq 0
\end{equation}

For reference

\begin{equation}\label{ent}
\Delta{S} = \int_{T_1}^{T_2} \frac{C_V}{T} \; dT
\end{equation}
\section*{Third Law}

If the entropy is constant, then the temperature doesn't change.

\begin{align*}
\label{third law}
\frac{\partial S_1}{\partial U_1} = \frac{\partial S_2}{\partial U_2} &\leftrightarrow T_1 = T_2 \; , \; \text{Thermodynamic} \;  \\
\frac{\partial S_1}{\partial V_1} = \frac{\partial S_2}{\partial V_2} &\leftrightarrow P_1 = P_2 \; , \; \text{Mechanical} \\
\frac{\partial S_1}{\partial N_1} = \frac{\partial S_2}{\partial N_2} &\leftrightarrow \mu_1 = \mu_2 \; , \; \text{Chemical}
\end{align*}

\section*{Multiplicity}

\begin{equation}
S = k \ln{\Omega}
\end{equation}

$$ S_{\text{total}} = S_A + S_B$$

Stirling's approximation

$$N! \approx N^Ne^{-N}\sqrt{2 \pi N}$$

For monatomic ideal gas


$$\Omega_N = \frac{1}{N!} \frac{V^N}{h^{3N}} \frac{2\pi^{P3N/2}}{(\frac{3N}{2} - 1 )!} \left ( \sqrt{2mU} \right ) ^{3N - 1}$$
Sackur-Tetrode equation
$$S = Nk\left [ \ln{ \left ( \frac{V}{N} \left ( \frac{4\pi mU}{3Nh^2} \right ) ^{3/2} \right )} + \frac{5}{2} \right ]$$

For mixing

$$\Delta{S}_{\text{mixing}} = -R[x \ln x + (1-x)\ln(1-x)]$$
$$G_{\text{mixing}} = G - T\Delta{S}_{\text{mixing}}$$

Stays in mixed state since

$$G_{\text{mixing}} < G$$
\section*{State variables and functions}

When we specify the equilibrium with a given value of some variable, that variable is a state variable.

$$ N, \; V, \; P , \; T , \; \mu, \; U, \; S , \; H, \; F, \; G $$ 

A state variable is a variable dependent on other variables.

Extensive variables are variables scaling with the system.

$$N, \; V, \; U , \; S , \; m, \; C_V $$

Intensive variables are variables which stay constant through system-scaling.

$$T, \; P, \; \mu, \; \rho$$

\section*{Energies}

\begin{center}
Thermodynamic Identity
\end{center}

\begin{equation}
\mathrm{d} U = T \mathrm{S} - P\mathrm{V} + \mu \mathrm{N}
\end{equation}

\begin{center}
Enthalpy
\end{center}

\begin{equation}
 H \equiv U + PV = H(S, P, N)
\end{equation}


\begin{center}
Helmotz
\end{center}

\begin{equation}
 F \equiv U -TS = F(T, V, N)
\end{equation}


\begin{center}
Gibbs
\end{center}

\begin{equation}
 G \equiv U -TS + PV = G(T, P, N)
\end{equation}


\begin{center}
Grand
\end{center}

\begin{equation}
 \Phi \equiv U -TS - \mu N = \Phi (V, T, N)
\end{equation}

\subsection*{Definitions}

The different energy-definitions gives us new ways to redefine variables.

\begin{equation}\label{T}
T = \left ( \frac{\partial U}{\partial S} \right )_{N,V} = \left ( \frac{\partial H}{\partial S} \right )_{N, P}
\end{equation}

\begin{equation}\label{P}
P = \left ( \frac{\partial S}{\partial V} \right )_{N, U} = -\left ( \frac{\partial F}{\partial V} \right )_{N, T} = -\left ( \frac{\partial \Phi}{\partial V} \right )_{\mu, T}
\end{equation}

\begin{equation}\label{mu}
\mu = - T \left ( \frac{\partial S}{\partial N} \right )_{U, V} = \left ( \frac{\partial H}{\partial N} \right )_{S, P} = \left ( \frac{\partial F}{\partial N} \right )_{V, T} = \left ( \frac{\partial G}{\partial N} \right )_{T, P}
\end{equation}

\begin{equation}\label{cv}
C_V= \left ( \frac{\partial U}{\partial T} \right )_{N, V}
\end{equation}

\begin{equation}
V = \left ( \frac{\partial H}{\partial P} \right )_{N, S} = \left ( \frac{\partial G}{\partial P} \right )_{N, T}
\end{equation}

\begin{equation}\label{entdef}
S = - \left ( \frac{\partial F}{\partial T} \right )_{N, V} = -\left ( \frac{\partial G}{\partial T} \right )_{N, P} = - \left ( \frac{\partial \Phi}{\partial T} \right )_{\mu, V}
\end{equation}

\begin{equation}
N = \left ( \frac{\partial \Phi}{\partial \mu} \right )_{V, T}
\end{equation}
\section*{Maxwells Relations}

For a state function $f$ with natural variables $x$ and $y$, we have the symmetry:

\begin{equation}
\frac{\partial^2 f}{\partial x \partial y} =  \frac{\partial^2 f}{\partial y \partial x}
\end{equation}

With $U$, $G$, $H$, and $F$ this gives us the relations

$$ \left( \frac{\partial T }{\partial V} \right )_S = - \left( \frac{\partial P }{\partial S} \right )_V$$
$$ \left( \frac{\partial T }{\partial P} \right )_S = \left( \frac{\partial V }{\partial S} \right )_P $$
$$ \left( \frac{\partial S }{\partial V} \right )_T = \left( \frac{\partial P }{\partial T} \right )_V $$
$$ -\left( \frac{\partial S }{\partial P} \right )_T = \left( \frac{\partial V }{\partial T} \right )_P $$

\section*{Second Law for F an G}
Hold $(N, V, T)$ constant

$$\Delta{F} \leq 0 \; .$$

Hold $(N,P,T)$ constant

$$ \Delta{G} \leq 0 \; .$$
\section*{Trouton's Rule}

$$\Delta{S}_{\text{vap}} = \frac{\Delta H }{T_b} \approx 85 \frac{J}{mol \cdot K} $$

\section*{Clasusis-Clapeyron Relation}

$$\ln{\frac{P1}{P2}} = - \frac{\Delta H}{R} \left ( \frac{1}{T_1} - \frac{1}{T_2} \right)$$

\section*{Osmotic Pressure}

Van't Hoffs formula

$$(P_1 - P_2) = \frac{kTN_B}{V} = \frac{RTn_B}{V}$$

\section{Constants}

Gas constant

$$R = 8.3145 \frac{J}{mol \cdot K}$$
$$k = 1.38064852 \cdot 10^{-23} \; m^2 \; kg \; s^{-2} K^{-1}$$

\end{document}
